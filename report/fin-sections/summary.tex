\begin{flushright}
  \TBdpt\\
  \TBfiliere\\
  \TBorient\\
  Student: \TBauthor\\
  Teacher in charge: \TBsupervisor\\
\end{flushright}

\vspace{0.6cm}

\begin{center}
  {\large Bachelor thesis \TBacademicYears \\[0.2cm]}
  {\TBtitle \\[0.5cm]}
\end{center}

\hrule
\vspace{0.5cm}

{Company name}

\TBindustryName

\vspace{0.5cm}

{\bfseries Publishable summary}

{
  NFC, Bluetooth, Wi-Fi and every common wireless communication protocol operate using radio waves. Emitting in the Radio Frequency (RF) spectrum requires complex analog components that, because of their nature, cannot be completely identical to one another. The inherent imperfections of these analog components are what makes RF fingerprinting possible. Indeed, this technique theoretically allows the identification of a device just through the analysis of its signal.

  This project's goal is to determine whether applying machine learning techniques to the problem of RF fingerprinting is effective for the NFC protocol, which is known to be vulnerable to relay attacks, for example. To do this, we first needed to build a dataset of NFC communications using Software Defined Radio (SDR) equipment. Then, we had to apply signal processing algorithms to process the data and finally, we built supervised deep learning systems in order to classify the tags.

  The project was very much an exploratory endeavour, so a lot of attention was given to the state of the art. Existing research in the field of RF Machine Learning (RFML) guided our experiments. After this analysis phase, a lot of work went into configuring and testing the SDR setup to acquire the data. We then wrote the software to manipulate this data and format it in order to feed it to our learning algorithms.

  In terms of results, our model can classify a signal segment from a known device up to around 85\% of the time, depending on the number of tags. The system is quite limited, and not yet scalable in any meaningful way, but it does answer some questions and open paths for future developments.
}
