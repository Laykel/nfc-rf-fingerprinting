\section{Initial project description}

Radio Frequency (RF) fingerprinting is a technique that allows the identification of radio transmitters (such as Internet of Things (IoT) devices) by analysing the spectrum of their transmissions. Indeed a device's spectrum is unique because of tiny imperfections in the manufacturing process of its analog components. This analysis can typically be performed using machine learning algorithms.

Near-Field Communication (NFC) technology is often used in access control and payment applications but many implementations are vulnerable to relay attacks. This type of attacks allows an attacker to relay messages between a reader and an NFC device without the knowledge of the device's owner, effectively convincing the reader it is communicating with the legitimate device. Research and tools that facilitate such attacks are publicly available.

The goal of this project is to determine if RF fingerprinting could be used as an authentication technique against relay attacks.

The main steps of this project are the following:

\begin{itemize}
  \item Build a simple lab setup with Software-Defined Radio (SDR) equipment to acquire signals between an NFC device and its reader
  \item Acquire RF spectrum data of various NFC devices
  \item Analyse the signals
  \item Classify the signals of the devices by using supervised machine learning classification techniques in order to differentiate trusted devices from attacker / relay devices
  \item Determine if this identification technique could be used as an authentication feature against relay attacks
\end{itemize}

As the receiving equipment (SDR) has an influence on the recorded signals, for this project we consider a single receiver to record the RF samples. Similarly, the lab setup should be built to provide an ideal low-noise \& low-interference environment to simplify the analysis phase.

The expected deliverables are the following:

\begin{itemize}
  \item A tool able to identify NFC devices by analysing the RF spectrum of their signals, at least in an ideal environment and with a small number of devices
  \item A detailed account of the steps taken and the setup used (as part of the report)
  \item An analysis of the results (as part of the report)
\end{itemize}

Collaboration with other researchers in this field is wished (EPFL, ElectroSense).
