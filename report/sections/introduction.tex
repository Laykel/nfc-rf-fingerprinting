\section{Introduction}

\subsection{Project description}
Small imperfections in the electromagnetic emissions of radio transmitters make it possible to identify them based only on the way they transmit. This is called Radio Frequency (RF) fingerprinting and it is possible thanks to tiny manufacturing imperfections in the devices' analog components. Using Software-Defined Radio (SDR) equipment, we can analyse this spectrum in order to extract the aforementioned differences and identify a device.

Such techniques can be used on any type of radio transmission: Bluetooth, Bluetooth Low Energy (BLE), WiFi, LTE (part of 4G mobile networks), etc. This project aims to use RF fingerprinting on NFC devices. Indeed, NFC is often used in access control and payment applications but many implementations are vulnerable to relay attacks. Spoofing the imperfections in an emitter's radio spectrum is close to impossible at the present time, since it is essentially a hardware signature. This is why a technique like the one described here would be a valuable additional security layer.

The goal of this project is to determine whether applying machine learning techniques to the problem of RF fingerprinting NFC devices could be used as an authentication technique, in order to prevent relay attacks. The first step to achieve this goal is to produce a dataset of raw NFC transmissions using SDR. Indeed, to our knowledge, there exists no available dataset of raw NFC captures.

\subsection{Context}
This project is conducted in the context of a bachelor thesis at HEIG-VD, the largest branch of the "University of Applied Sciences Western Switzerland" (HES-SO).

\begin{itemize}
  \item Department: Information and communication technologies
  \item Faculty: Information technology and communication systems
  \item Orientation: Software engineering
\end{itemize}

It was proposed by Mr Joël Conus of \TBindustryName.

\subsection{Document description}
This intermediary report marks the middle of the project. Because of this, it is firmly anchored in the analysis and conception phases, which means much of what is presented is subject to change in the second half of the project.

Nevertheless, this document describes the research done while studying the state of the art. It then presents the acquisition setup and the results it brought, before showing the steps undertaken to validate the captured signals through decoding. Finally, it showcases the first conception ideas and decisions made for the learning model, in light of our study of the state of the art.
